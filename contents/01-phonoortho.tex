\section{Phonology}%

\subsection{Consonant}%
Krewi consonants are divided into 11 contrastive units as depicted in the matrix provided below. Phonemes in brackets are borrowed-origin.

\begin{longtblr}[
		halign = c,
	]{
		rowhead = 1,
		vline{2} = {0.5pt},
		hline{2} = {0.5pt}
	}
	\toprule[1pt]
	                   & \small{labial} & \small{lingual (tip)} & \small{lingual (dorsum)} & \small{glottal} \\
	\small{nasal}      & m              & n                     &                          &                 \\
	\small{obstruents} &                & d dz                  & (dʒ)                     & x ħ (k)         \\
	\small{emphatics}  &                & ts                    & ɟʝ (tɕ)                  & kʷ kʲ           \\
	\small{continuant} & w [w ʝʷ]       & l r                   & j [j ɟ]                                    \\
	\bottomrule[1pt]
\end{longtblr}

Observed allophones are as follows.

\begin{itemize}
	\item Nasal /n/ can be further analyzed into two phonemes. The onset /n/ is a fairly stable phoneme which has no allophone. Meanwhile, coda /n/ allophones surface homorganically depending on the context. Coda /n/ shift the articulatory point to match the next, usually less sonorous, onset. Such phenomenon are, namely, dorsal nasal (e.g. \mbox{ɣəɲ.ɟaj}) and glottal nasal (e.g \mbox{tsɤŋ.ɡaw}). In some cases, final /n/ may surfaces the pausal form derived from the retracted point of the onset (e.g. \mbox{wɛ.raɲː}).
	\item Native obstruents, in exception of /ħ/, realize in many degree of strength. Beside the condition whichever plain [d dz x] occur, tense [t k] are only realized before laterals [l r]. Lax [ɾ z h\textasciitilde Ø] are only realized in intervocalic conditions. Labialized [tʷ ɡʷ] are only realized in the stop + semivowel [w] sequence. Meanwhile, the presence of the semivowel [j] turns obstruents into supra-tense emphatics [ɟʝ kʲ], but despite the relatedness, they fall into different phonemes.
	\item Non-native Peninsular dorsals /dʒ tɕ/ are pronounced by analogy of [dz ɟʝ] with retracted alticulatory point.


\end{itemize}

\subsection{Vowel}
Krewi vowels had four native phonemic monophthongs /i ə a ɤ/ and one long /eː/ which exists in loanwords.

\begin{figure}[h]
	\centering
	\begin{tikzpicture}
		\coordinate (hf) at (0,2);
		\coordinate (hb) at (2,2);
		\coordinate (lf) at (1,0);
		\coordinate (lb) at (2,0);
	\end{tikzpicture}
	\caption{Mapped vowels}
\end{figure}

\subsubsection{Weak and strong vowels}
In contrast to the relatively stable /i a/, the vowel /ə/ is considered weak and pronounced with a relatively open quality and may be approximated as [ɛ]. In stressed environment, the vowel /ə/ is realized as low as [a]. On the other hand, the vowel /ɤ/ can be realized much back and acquired the roundedness to [u] when influenced by high vowel /i/.

\subsubsection{Adoption of Peninsular /e/}
Many Peninsular languages have mid-front vowel /e/. Krewi does not have this sound as a native phoneme, and speakers tend to emulate such sound more lower as [ɛː], using analogy of finer-stressed pattern of [ə]>[a], with gemination is thought as an overcompensation. Less-educated speaker and in casual speech often uses [ɪ] as the realization of the phoneme.

\subsubsection{High and low back vowel}
The back vowel phoneme /ɤ/ occupies wide range of realization, mainly influenced by the surrounding vowels and tend to assimilate the realization with the front vowels before it. Such realizations are: after /i/, it becomes

\subsection{Romanization scheme}%

Empty cells are conditions deemed impossible in Krewi.

\begin{longtblr}[
		halign = c,
		note{a} = {Only in intervocal context},
		note{b} = {Only after back coda viz. [ŋ]},
		note{c} = {Only as pausa},
	]
	{
		rowhead = 1,
		cells = {halign=c}
	}
	\toprule
	\small{archetype} & \small{lax}          & \small{tense} & \small{\_r} & \small{\_w} & \small{\_\{ɪ, ɛ, \sc{velar}\}} & \small{back}       & \small{long}   \\
	\midrule
	ɣ ⟨gh⟩            & Ø, h ⟨h⟩\TblrNote{a} & ɡː ⟨gg⟩       & x ⟨k⟩       & kʷ ⟨k⟩      & kʲ ⟨ky⟩                        & k ⟨gh⟩\TblrNote{b} &                \\
	ħ ⟨hh⟩            &                      & k ⟨hh⟩        &             &             &                                &                    &                \\
	d ⟨d⟩             & ɾ ⟨d⟩                & tsː ⟨tt⟩      & t ⟨t⟩       & tˠ ⟨t⟩      & tʃ ⟨c⟩                         &                    &              & \\
	z ⟨z⟩             &                      &               &             &             & dz ⟨z⟩                         &                    &              & \\
	ɟ ⟨j⟩             & j ⟨y⟩                & ɟʝ ⟨jj⟩       &             &             &                                &                    &              & \\
	w ⟨w⟩             &                      &               &             &             & ʝʷ ⟨zz⟩                        &                    & wː ⟨ww⟩        \\
	n ⟨n⟩             &                      & ɲː ⟨nn⟩       &             & w̃ ⟨w⟩       & ɲ ⟨ny⟩\TblrNote{c}             & ŋ ⟨ng⟩             & nː ⟨nn⟩        \\
	l ⟨l⟩             &                      &               &             &             &                                &                    & lː ⟨ll⟩        \\
	r ⟨r⟩             &                      &               &             &             &                                &                    & rː ⟨rr⟩        \\
	m ⟨m⟩             &                      &               &             &             &                                &                    & mː ⟨mm⟩        \\
	\bottomrule
\end{longtblr}

