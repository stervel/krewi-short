
\section{Lexical Categories}
\subsection{Open classes}
There are some lexical roots that can appear in a nominal frame or in a verbal frame without any clear morphological changes. Examples are given in (\nextx).

\pex
\a \begingl
\gla ghehheye \underline{zeuraw} //
\glb own-I broom //
\glft `I own a broom' //
\endgl
\a \begingl
\gla \underline{zeuraw} jaheuw ghemalar //
\glb sweep I.\sc{agw} this.place //
\glft `I have to sweep this place' //
\endgl
\a \begingl
\gla \underline{zeuraw} zzihemer hhije méhe ralaw //
\glb sweep swing he he.\sc{attr} tail.\sc{cjt} //
\glft `His tail wags broom-likely' or `his tail wags in the manner of broom sweeping' //
\endgl
\xe

\noindent Seen above, the word \emph{zeuraw} is used nominally in (\lastx a) and verbally in (\lastx b) in the same form. Based solely on these sentences, it can be deduced that the task of determining whether a root is a noun or a verb is not always straightforward. However, establishing a distinction between Krewi nouns and verbs is relatively simple.

\subsubsection{Noun category}
One of a route to identify a noun is that by using specifier as a diagnostics. Morphologically, only nouns can be attached with \emph{mi-} `same \ldots'.\todo

\ex \vtop{\halign{%
		#\hfil&& \qquad #\hfil\cr
		\emph{zeuraw}& `broom' & \emph{mezeuraw}& `same broom' \cr
		\emph{leheye}& `person' & \emph{miléhaye}& `same person' \cr
		\emph{gharew}& `mother' & \emph{méharew}& `same mother' \cr
		\emph{lemer}& `walk' & {?\emph{milémar}}& `same journey' or `same pathway' \cr
		\emph{cenay}& `read' & \emph{*micéney} \cr
		\emph{hhejeun}& `consume' & \emph{*mihhéjeun} \cr
		\emph{nahay}& `bad' & \emph{ménahay} \cr
	}}\xe



