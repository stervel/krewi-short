\section{Noun Morphology}%
In general, nouns can be a single morpheme or consist of multiple morphemes. Multiple morpheme nouns are derived from single morpheme nouns or some other form classes through three basic morphological processes, namely affixation, reduplication, and various allomorph that constitutes suppletive processes.

\subsection{Affixation}%
Affixation processes in Krewi includes prefixes, suffixes, and infixes. Many of those often multifunctional and the same configuration often bleeds into other classification, resulting many shared morphemes used in many conditions but different effects on the stem. A prime example of this is the infix \emph{-ig-} which, depending on the root, can derive root words into many conditions.

\pex
\a \vtop{\halign{%
#\hfil&& \quad #\hfil\cr
we-{\sc v} &\cr
\emph{daran} `to announce' & \emph{wedaran} `announcement' \cr
\emph{cenay} `to read' & \emph{wecanay} `reading material' \cr
\emph{ghereuy} `to want` & \emph{wahereuy} `desire` \cr
}}
\a \vtop{\halign{%
#\hfil&& \quad #\hfil\cr
we-{\sc num} &\cr
\emph{ghi} `one' & \emph{wéhi} `first' \cr
\emph{zawwar} `two' & \emph{wazawwar} `second' \cr
\emph{gherceuw} `eight' & \emph{waheurcuw} `eighth' \cr
}}
\xe


But if the verbal root is inherently patientive, the process results a habitual or frequentative verb. \todo

\ex\vtop{\halign{%
		#\hfil&& \qquad #\hfil\cr
		\emph{ghelawe} `to melt' & \emph{ghihelew} `melt easily, frequently' \cr
		\emph{jameur} `to go home' & \emph{jehameur} `go home often' \cr
	}}\xe


